\subsection{LHC}
\subsection{ATLAS}
\subsection{Luminosity detectors}
\subsection{Inner detector}
\subsection{Calorimeter}
Calorimeters provide accurate measurements of the energies and positions of electrons, photons, and jets as well as of the missing transverse energy. Calorimetric measurements are also crucial to particle identification, serving to distinguish electrons and photons from jets, and also helping to identify hadronic decays of tau leptons. 
The major components are the liquid argon (LAr) barrel (EMB) and endcap (EMC) electromagnetic (EM) calorimeters covering $|\eta| <$ 3.2, the tile scintillator hadronic barrel calorimeter covering $|\eta| <$ 1.7, the LAr hadronic endcap calorimeter (HEC) covering 1.5 $< |\eta| <$ 3.2, and the LAr forward calorimeter (FCAL) covering 3.1 $< |\eta| <$ 4.9. The electromagnetic calorimeters use lead absorbers and a LAr ionization medium, and are contained in three separate cryostats: one for the barrel and two for the endcaps. 
The calorimeters have an accordion geometry that provides full φ symmetry without azimuthal cracks. They are segmented longitudinally into three layers (called strips, middle, and back). The middle layer contains around 80\% of the energy of an electromagnetic shower. The cell size $\Delta\eta \times\Delta\phi$ is 0.025$\times$0.025 in the middle layer and 0.003$\times$ 0.1 in the strips in the barrel calorimeter (the cells are larger at higher $|\eta|$), allowing very precise η measurements of incident particles. A presampler (PS) covers the region $|\eta| <$ 1.8 to improve the energy measurement for particles that start showering before entering the calorimeter. Plastic scintillator tiles are placed between the cryostats in order to recover some of the energy that is lost in dead material in this region. 
Wrapped around the LAr calorimeter cryostats is the barrel hadronic calorimeter. It uses iron absorbers interleaved with plastic scintillator tiles. 
The central barrel portion covers $ |\eta| < $ 1.0; two extended barrel calorimeters cover 0.8 $ < |\eta| < $ 1.7. The 68 cm gaps between the central and extended barrels are also instrumented with plastic scintillator sheets. The endcaps of the hadronic calorimeter again use the liquid argon technology, due to the high radiation doses experienced in the forward regions. 
For 1.5 $ < |\eta| < $ 3.2, copper plate absorbers are used, and the calorimeters are installed in the same cryostats as the EM endcaps. 
The FCAL, covering $ |\eta| > $ 3.1, consists of rod-shaped electrodes embedded in a tungsten matrix. The cell sizes in the hadronic calorimeters are larger than in the electromagnetic calorimeters; ranging from 0.1x0.1 to 0.2x0.2. The tile calorimeter is divided into three longitudinal layers, while the HEC has four layers. The FCAL consists of three modules in depth. 
Noise in the calorimeter comes from two principal sources. The first is from the readout electronics. The second is called “pile-up” noise, and arises from extra interactions that can either be overlaid in the same beam crossing with the primary interaction or occur during crossings that are close in time to that of the primary interaction (as the response time of the calorimeter is longer than the 25 ns interval between crossings). 
Incoming particles usually deposit their energy in many calorimeter cells, both in the lateral and longitudinal directions. Clustering algorithms are designed to group these cells and to sum the total deposited energy within each cluster. These energies are then calibrated to account for the energy deposited outside the cluster and in dead material. The calibration depends on the incoming particle type; the calibration for electrons and photons is described in Ref. , and the calibration for jets in Ref. .

\subsection{Muon spectrometer}
The MS is the outermost of the ATLAS sub-detectors: it is designed to detect charged particles in the pseudorapidity region up to  $ |\eta| $ = 2.7, and to provide momentum measurement with a relative resolution better than 3\% over a wide $p_{T}$ range and up to 10\% at pT $\approx$ 1 TeV. The MS consists of one barrel part (for $ |\eta| <$ 1.05) and two end-cap sections. A system of three large superconducting air-core toroid magnets provides a magnetic field. Triggering and $\eta$, $\phi$ position measurements, with typical spatial resolution of 5−10 mm, are provided by the Resistive Plate Chambers (RPC, three doublet layers for $ |\eta| <$ 1.05) and by the Thin Gap Chambers (TGC, three triplet and doublet layers for 1.0 $<  |\eta|  <$ 2.4). Precise muon momentum measurement is possible up to $ |\eta| $ = 2.7 and it is provided by three layers of Monitored Drift Tube Chambers (MDT), each chamber providing six to eight $ |\eta| $ measurements along the muon track. For $ |\eta|  >$ 2 the inner layer is instrumented with a quadruplet of Cathode Strip Chambers (CSC) instead of MDTs. The single hit resolution in the bending plane for the MDT and the CSC is about 80 $\mu$m and 60 $\mu$m, respectively. Tracks in the MS are reconstructed in two steps: first local track segments are sought within each layer of chambers and then local track segments from different layers are combined into full MS tracks.
\subsection{Magnet system}
\subsection{Trigger system}
\subsection{Data acquisition}