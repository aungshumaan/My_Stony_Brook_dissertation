\subsection{$\ttbar$ modeling}
$\ttbar$ is one of the bigger  backgrounds. It has been modeled by the Powheg generator which calculates the matrix element 
at next to leading order (NLO). The parton shower has been calculated with Pythia. 

Previous ATLAS analyses have shown that Powheg-Pythia samples do not model experimental data perfectly, as evidenced by the unfolded differential top quark pair cross-section measurement. Following the conclusions of the study, two reweighting functions were derived so that the MC sample agrees with unfolded data for the $p_{T}$ of the parton-level $\ttbar$ and the $p_{T}$ of the top-quark (not anti-top). These weights are applied as additional event weights. The numerical value of the weights and more details can be found in~\cite{TWikiReweighting}. This procedure has been followed by previous analyses ~\cite{Aad:2015gra}.


%The modeling has been checked by looking at the data-MC agreement in the top-quark control regions.
%Data-MC comparisons are shown in Figures~\ref{fig:dataMC_topCR_resolved_mp}-\ref{fig:dataMC_topCR_FJ_em}, for the top-quark control regions for the loose VBS cuts.  
\subsection{QCD fit}
There is a small background contribution from QCD multijet processes.  These backgrounds come primarily from jets misreconstructed as electrons,
or from leptons originating from heavy-flavor decays inside jets. For simplicity all of these sources will be referred to as ``fake'' leptons.
Because of the very high cross-section and low fake rate of these processes, and because of the difficulty in modeling the fake-rate, 
this background is difficult to model with MC. Therefore, it has been estimated using a data-driven method. 

A fake-lepton-enriched region can be constructed by modifying the identification criteria of the leptons, to 
create ``bad'' lepton candidates.  Bad electrons are required to pass the medium++ requirement but fail the tight++ one. 
For bad muons, the cut on the significance of transverse component of impact parameter, $|\dzsig|$, is inverted with 
respect to the good muon definition, i.e. bad muons must satisify $|\dzsig|>3$. In order to improve the statistics and purity 
of the fake-enriched region, the isolation requirements on the bad lepton  are also modified with respect to the good lepton
definition.   We require bad leptons to have $\Sigma E_{T}$Cone30/$p_{T}$ > 0.04 and $\Sigma p_{T}$Cone30/$p_{T}$ <0.5.
