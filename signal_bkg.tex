\subsection{$\ttbar$ modeling}
$\ttbar$ is one of the bigger  backgrounds. It has been modeled by the Powheg generator which calculates the matrix element 
at next to leading order (NLO). The parton shower has been calculated with Pythia. 

Previous ATLAS analyses have shown that Powheg-Pythia samples do not model experimental data perfectly, as evidenced by the unfolded differential top quark pair cross-section measurement. Following the conclusions of the study, two reweighting functions were derived so that the MC sample agrees with unfolded data for the $p_{T}$ of the parton-level $\ttbar$ and the $p_{T}$ of the top-quark (not anti-top). These weights are applied as additional event weights. The numerical value of the weights and more details can be found in~\cite{TWikiReweighting}. This procedure has been followed by previous analyses ~\cite{Aad:2015gra}.


%The modeling has been checked by looking at the data-MC agreement in the top-quark control regions.
%Data-MC comparisons are shown in Figures~\ref{fig:dataMC_topCR_resolved_mp}-\ref{fig:dataMC_topCR_FJ_em}, for the top-quark control regions for the loose VBS cuts.  
\subsection{QCD fit}